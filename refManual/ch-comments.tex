\chapter{Comments}
\index{Comments}

It's always a good idea to include comments in your programs, and RTB
provides 2 methods to help you do this.

Firstly, there is the traditional {\tt REM} statement. Short for
``Remark''.  Secondly there is the {\tt //} statement which is
common in many other programming languages.

{\tt REM} must appear at the start of a program line, but {\ //} may
appear anywhere in a line - and anything after the {\tt //} is ignored.

Examples:
\begin{verbatim}
  100 REM This is a demonstration of comments
  110 //
  120 REM // on its own can be used to separate program sections.
  130 //
  140 // The line below is allowed:
  150 LET test = 42 // Set test to 42
  160 //
  170 // But the line below this is not allowed:
  180 LET test = 42 REM Set test to 42
\end{verbatim}
