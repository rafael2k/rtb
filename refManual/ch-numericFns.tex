\chapter{Numerical Functions}
\index{Numerical Functions}

\begin{description}
\item[{\tt RND (range)}]\index{RND}
This function returns a random number based on the value of {\tt range}.
If {\tt range} is zero, then the last random number generated is returned,
if {\tt range} is 1, then a random number from 0 to 1 is returned,
otherwise a random number from 0 up to, but not including {\tt range}
is returned.
\item[{\tt SIN (angle)}]\index{SIN}
Returns the sine of the given {\tt angle}.
\item[{\tt ASIN (x)}]\index{ASIN}
Returns the arc sine of the supplied argument {\tt x}
\item[{\tt COS (angle)}]\index{COS}
Returns the cosine of the given {\tt angle}.
\item[{\tt ACOS (x)}]\index{ACOS}
Returns the arc cosine of the supplied argument {\tt x}
\item[{\tt TAN (angle)}]\index{TAN}
Returns the tangent of the given {\tt angle}.
\item[{\tt ATAN (x)}]\index{ATAN}
Returns the arc tangent of the supplied argument {\tt x}
\item[{\tt ABS (x)}]\index{ABS}
Returns the absolute value of the supplied argument {\tt x} ie. if the
argument is negative, make it positive.
\item[{\tt EXP (x)}]\index{EXP}
Returns the value of $e^x$
\item[{\tt LOG (x)}]\index{LOG}
Returns the natural logarithm of {\tt x}
\item[{\tt SQRT (x)}]\index{SQRT}
Returns the square root of {\tt x}
\item[{\tt SGN (x)}]\index{SGN}
Returns -1 if the number is negative, 1 otherwise. (Zero is considered
positive)
\item[{\tt INT (x)}]\index{INT}
Returns the integer part of {\tt x}
\item[{\tt HASH (string, modulo)}]\index{HASH}
Returns a hash value based on the {\tt string} and {\tt modulo}. The
value returned is between 0 and {\tt modulo}. This function is used
internally in RTB for the associative array indexing. The hash function
is based on the function in {\em The Practice of Programming (HASH TABLES,
pg. 57)}
\item[{\tt MAX (x, y)}]\index{MAX}
Returns the larger of {\tt x} or {\tt y}
\item[{\tt MIN (x, y)}]\index{MIN}
Returns the smaller of {\tt x} or {\tt y}
\item[{\tt VAL (string\$)}]\index{VAL}
Returns the number represented by {\tt string\$}. e.g. "1234" would
return the number 1234.
\item[{\tt ASC (string\$)}]\index{ASC}
Returns the ASCII value represented by the first character of {\tt
string\$}. e.g. "A" would return 65. It is the opposite of the {\tt
CHR\$} function.
\item[{\tt LEN (string\$)}]\index{LEN}
Returns the number of characters in {\tt string\$}.
\end{description}
