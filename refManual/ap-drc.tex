\chapter{DRC - Drogon Remote Control Programming}
\index{DRC}
RTB supports the Drogon Remote Control protocol which is used to connect
to devices which interpret the DRC protocol. DRC is typically used with
small micro-controllers such as the Arduino and it allows the control of
the digital and analogue input/output pins to be brought to RTB.

\section{DrcOpen}
\index{DRC!DrcOpen}
This opens a connection to a DRC compatible device and makes it
available for our use. It takes the name of the device as an argument
and returns a number (the {\em handle}) of the device. We can use this
handle to reference the device and allow us to open several devices at
once. Some implementations may have IO devices with fixed named. Example:
\begin{verbatim}
  100 arduino = DrcOpen ("/dev/ttyUSB0")
  110 gpio = DrcOpen ("RPI")
\end{verbatim}
On the Raspberry Pi, you may use the string: {\tt "RPI"}, {\tt RPI-GPIO}
or {\tt RPI-SYS} as the device to open and this will open the on-board
GPIO pins in native wiringPi, GPIO or Sys modes.

\section{DrcClose}
\index{DRC!DrcClose}
This closes a connection to a DRC device and frees up and resources used
by it. It's not strictly necessary to do this when you end your program,
but it is considered good practice.
\begin{verbatim}
  120 DrcClose (arduino)
\end{verbatim}

\section{DrcPinMode}
\index{DRC!DrcPinMode}
This configures the mode of a pin on the remote DRC device. It takes
an argument which specifies the mode of the pin - input, output or PWM
output. Other modes may be available, depending on the device and its 
capabilities. Note that not all devices support all functions.
\begin{verbatim}
  220 DrcPinMode (arduino, 4, 0)
\end{verbatim}
In this example, we're setting pin 4 to be used for input. The modes are;
\begin{description}
\item[0] Input
\item[1] Output
\item[2] PWM Output
\end{description}

\section{DrcDigitalRead}
\index{DRC!DrcDigitalRead}
This function allows you to read the state of a digital pin on the remote
device. You may need to set the pin mode beforehand to make sure it's
configured as an input device. It will return {\tt TRUE} or {\tt FALSE}
to indicate an input being high or low respectively.
\begin{verbatim}
  200 DrcPinMode (arduino, 12, 0) // Set pin 12 to input
  210 if DrcDigitalRead (arduino, 12) THEN PROC ButtonPushed
\end{verbatim}

\section{DrcDigitalWrite}
\index{DRC!DrcDigitalWrite}
This procedure sets a digital pin to the supplied value - 0 for off or
1 for on. As with {\tt DigitalRead}, you may need to set the pin mode
beforehand.
\begin{verbatim}
  310 DrcPinMode (arduino, 2, 1) // Set pin 2 to output mode
  320 DrcDigitalWrite (arduino, 2, 1) // Set output High (on)
  330 Wait (1)
  330 DrcDigitalWrite (arduino, 2, 0) // Set output Low (off)
\end{verbatim} 

\section{DrcAnalogRead}
\index{DRC!DrcAnalogRead}
This function reads an analog channel and returns the result. The value
returned will depend on the hardware you're connected to - for example
the Arduino will return a number from 0 to 1023 representing an input
voltage between 0 and 5 volts. Other devices may have different ranges.
\begin{verbatim}
  250 voltage = DrcAnalogRead (arduino, 4) / 1023 * 5  // Get voltage on pin 4
\end{verbatim}

\section{DrcPwmWrite}
\index{DRC!DrcPwmWrite}
This procedure outputs a PWM waveform on the selected pin. The pin must
be configured for PWM mode beforehand, and depending on the device you
are using, then not all pins on a device may support PWM mode. The value
set should be between 0 and 255.
\begin{verbatim}
  310 DrcPinMode (arduino, 11, 2) // Set pin 11 to PWM output mode
  320 DrcPwmWrite (arduino, 11, 200)
\end{verbatim} 
