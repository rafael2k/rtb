\chapter{Introduction}
\section{Audience}
\index{Audience}
This manual is aimed at people who may already have a familiarity with
BASIC or other high-level languages. It's mainly a reference manual
for the language, but that are a few examples and tutorials.

The manual will assume that you understand most common principles
of computing and know concepts like volatile and non-volatile storage
and so on and how to type some commands into your computer.

And although it's a reference manual, there are one or two little
examples and tutorials along the way for you to try out.

\section{Conventions Used in this Manual}
\index{Conventions}
Keeping things simple is the aim here, so there are only really 3
things to look out for\footnote{OK, Possibly four as there may be
the occasional footnote at the bottom of a page, so look out for them
too.}. The first is that anything you may need to type into a computer
running RTB, or anything it prints will be in this {\tt typewriter}
style font. The second is that anything of some importance, such as a
name of an algorithm is emphasised {\em like this}.

The \meek final thing to look out for is a warning, or just something
that may be important to remember. It's represented by a warning
exclamation point in a box to the side of the text. As demonstrated in
this paragraph.\newpage
\index{Warning}

If you know what you're doing, here are some brief differences between
RTB and a ``classic'' BASIC:
\begin{itemize}
\item RTB does not allow multiple statements on one line.
One statement per line only.
\item Variable names can be of almost any length and upper/lower case
is significant.
\item {\tt IF} statements must have {\tt THEN} (So no {\tt IF\dots\
num = 4}, or {\tt IF\dots\ GOTO 10}, it must be the full {\tt IF\dots\
THEN GOTO 10}, or {\tt IF\dots\ THEN a = 7}, etc.)
\item In {\tt FOR} loops, there is no {\tt NEXT} instruction -- it's
replaced by the {\tt CYCLE\dots REPEAT} construct.
\item Named procedures and functions. (Which can be called recursively)
\item Local variables inside user-defined procedures and functions.
\item A single looping construct ({\tt CYCLE\dots REPEAT}) which can be
modified with {\tt FOR}, {\tt WHILE} and {\tt UNTIL} constructs.
\item {\tt BREAK} and {\tt CONTINUE} as part of the looping construct.
\item Arrays start at zero and go up to and include the number in the
{\tt DIM} statement. ie. the {\tt DIM} statement specifies the size of
the array plus one.
\end{itemize}
Additionally (where supported) the graphical systems may well be new or
different to you. RTB incorporates simple block/line graphics as well
as turtle graphics using a variety of angle modes (degrees, radians
and clock).
