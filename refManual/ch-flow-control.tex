\chapter{Flow Control}
\index{Flow Control}
Programs are normally executed in line-number order, one after the other.
There are many ways in which we can alter the flow of our programs, but
the traditional BASIC ones are the {\tt GOTO} and {\tt GOSUB}
instructions.

{\tt GOTO}\index{GOTO} as its name implys causes program execution to go to the line
number listed after the {\tt GOTO} instruction.
\begin{verbatim}
  120 GOTO 315
\end{verbatim}

{\tt GOSUB}\index{GOSUB}\footnote{Go to Subroutine} allows a program to
temporarily jump to a new point and then, upon execution of the {\tt
RETURN} statement, flow resumes at the statement after the {\tt GOSUB}
instruction.

{\tt GOSUB} is designed to be used to allow a piece of code to be executed
over again from different parts of the main program.

Subroutines can call other subroutines and the number of subroutines that
can be called is limited only by the memory capacity of the computer -
some of which is required to keep track of where to return to.

Here is an example -- It's code to print my name:
\begin{verbatim}
  500 REM Simple subroutine to print my name
  510 PRINT "Gordon Henderson"
  580 RETURN
\end{verbatim}
Then, anywhere we need to print my name:
\begin{verbatim}
  100 PRINT "Hello ";
  110 GOSUB 500
  120 PRINT "Good to meet you ";
  130 GOSUB 500
  140 GOTO 100
\end{verbatim}

Subroutines are a good was of saving and re-using code however there
are more modern ways of doing this that removes the need to keep track
of line-numbers.

\section{GOTO and GOSUB -- The great controversy}
\index{GOTO!Controversy}
\index{GOSUB!Controversy}
The use of {\tt GOTO} and {\tt GOSUB} is highly debated and both {\tt
GOTO} and {\tt GOSUB} should be considered deprecated in their use. It
is possible to write fully functional RTB programs without using either
these instructions.

The original version of BASIC did not provide any more means to
control the flow of your code, however newer programming languages have
evolved which do help here, and in some areas the use of the {\tt GOTO}
instruction has been eliminated entirely!

However don't fix the idea in your head that all GOTOs are bad, we
need to balance things up and note that not everyone thinks that GOTOs
are bad -- myself included. Very occasionally, a GOTO can get you out
of a situation in a more elegant manner than some of the constructs
you'll read about next, so don't be afraid of the GOTO, but instead
respect\index{GOTO!Respect} it, try not to use it, but if you have to
use it, then use it well!
