\chapter{The command-line}\index{Command line}
The command line is the general term for typing commands and 
program lines into the system, however it has a few features to
make your life easier.

\section{Editing}\index{Command line!Editing}
As you type characters into the system, you may make mistakes. To correct
them you can use several differenet keys. The important one is probably
the {\tt Backspace} key. Usually a big key to the top-right of the main
keyboard with an arrow pointing to the left. This will erase characters
you've typed and move the cursor to the left, however there are more
efficient ways to deal with the like you're typing noted below.

Some of  the command characters lsited below you access with the {\tt
Control} key. This is labelled as {\tt Ctrl} on most keyboards and
works like the {\tt Shift} keys in that you push it, keep it pushed,
then type another character before releasing them both.

\begin{description}
\item[{\tt Ctrl-A}, {\tt Ctrl-E}:] Move the cursor to the start or the
end of the line respectively. You can also use the {\tt Home} and {\tt End}
keys if your keyboard has them.
\item[{\tt $\leftarrow$ (Left)} and {\tt $\rightarrow$ (Right)}:] arrow
keys (not to be confused with the {\tt Backspace} key) move the cursor
one character to the left or right of the typed line.
\item[{\tt Ctrl-D}:] Delete the character under the cursor. You can also
use the {\tt Del} or {\tt Delete} key if your keyboard has one.
\item[{\tt Backspace}:] Deletes the character to the Left of the cursor)
\item[{\tt Ctrl-S}:] Swap the character under the cursor with the character
immediately to the right. Handy for those who type the as teh like me\dots
\item[{\tt Ctrl-F}:] Find the next occurance of the next character typed.
E.g. {\tt Ctrl-F} followed by {\tt G} will make the cursor jump to
the next {\tt G} in the line. {\tt Ctrl-F} followed by another {\tt
Ctrl-F} will repeat the last find. 
\item[{\tt Esc}:] Abandon entering this entire line.
\end{description}

\section{History}
\index{Command line!History}
The RTB command line remembers the past 50 lines that you type in,
and you can use the $\uparrow$ ({\tt Up}) and $\downarrow$ ({\tt Down})
arrow keys on your keyboard to scroll through past things you've typed
in. This will enable you to quickly fix a mistake in a line already
in the system, or a line you typed in which remoted an error after you
pressed the {\em Enter} key.

\section{Editing program lines}
\index{Command line!Edit program lines}
If you spot a mistake in a program line and you can't find it in the
history, then you can enter the {\tt ED} command followed by the line
number. E.g. {\tt ED 10} will then present line 10 as if you had
re-typed it, but not yet pressed {\em Enter} so you can change the line
as required.
