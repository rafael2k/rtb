\chapter{The computer}

\section{What is a computer?}

At the simplest level, we have a processor which controls everything,
memory which holds our programs and data, a storage device which will
remember our programs and data when the system is turned off, and a means
to get data into the computer and to let us see what data and programmes
are in the computer.

In a little more detail, we have:

\begin{description}

\item[Processor] At the heart of the system is the processor. This
has many different names -- for example, it's often called the CPU
(Central Processing Unit) and sometimes known as a microprocessor. Some
computers have more than one processor and these are often referred to as
``cores''. For now, just thinking about a single processor is sufficient
for our needs. Processor speeds are measured in a variety of ways,
but we often refer to the base speed in cycles per second, or Hertz
(Hz) 3 billion Hz (3,000,000,000Hz) is common today, but in the 1970s,
1 million Hz was as fast as they went.
\index{Processor}

We often use names and letters to represent
big numbers, so a million is called ``mega'' and a billion is often called
``giga'' -- so 3 billion Hz is is three gigahertz, or 3GHz and so on. Fast
forward to {\em Appendix A} for details of numbering names and sizes.

\item[Memory] The processor needs somewhere to store programs and data
and this is called the memory. It's sometimes called RAM, short for
Random Access Memory -- random because the processor can access any
part of it very quickly at random, rather than having to start at the
beginning and work through sequentially. (Very old computers sometimes
had to do this) We talk about memory capacity in terms of ``bytes''. A
byte is the amount of memory required to store one character or one
single digit. Modern computers have or millions, or even billions of
bytes of memory. In the 1970s, computers only have thousands of bytes of
memory. Again, fast forward to {\em Appendix A} for details of numbering
names and sizes.
\index{Memory}

\item[Storage] Computer memory is ``volatile'' -- meaning that when you
turn the computer off the memory will forget what was contained in it,
so we need some form of non-volatile storage. This is often a disk drive
made up from one of more spinning disks using magnetism to record the
data, or sometimes it's made from a type of memory that is designed to
not forget it's contents when switched off. This is often called ``flash''
memory and comes in many formats -- plug-in devices, cards and so on.
\index{Storage}\index{Volatile}\index{Non-Volatile}

You may think it's a good idea to make all memory remember all the time,
but there are trade-offs to be had. The main computer memory is fast. Very
fast! The non-volatile memory on disks and flash devices are much slower
by comparison, however they are also cheaper -- sometimes by a factor of
1000 times or more. The trade-off is cost against speed.

\item[Input and Output] We have a processor, memory and storage -- the
other things we need are collectively called Input/Output devices,
or just ``IO''. These are usually a screen to see what the computer
is doing and a keyboard and possibly a mouse to give commands to the
computer. Other devices might include a printer or plotter, a scanner
and a rocket engine should be you lucky enough to be writing software
for the next Mars mission\dots
\end{description}
\index{Input} \index{Output}

\section{What is a computer program?}
\index{Computer Program}

Quite simply, a computer program is a set of instructions that the
computer will execute one after the other to perform a given task.

\noindent
Here is an example of a human program: 
\index{Human Program}

\setcounter{teaMakePage}{\value{page}}
\begin{enumerate}
\item GO TO Kitchen.
\item IF Kettle is empty THEN fill kettle with water.
\item Turn kettle on.
\item Get Mug. Get teabag. Put teabag into mug.
\item Wait for kettle to boil.
\item Pour boiling water into mug.
\item Wait for tea to infuse.
\item Remove teabag from mug.
\item Add milk to mug.
\item Enjoy mug of tea.
\item IF more tea required THEN GO TO step 2.
\end{enumerate}

Computer programs are not really much different from this, although
they're unlikely to make you a good mug of tea\dots Also, just as there
are many ways to make a mug of tea, then there are just as many ways to
write a computer program!

Program instructions tell the computer to do many different things -
for example, one instruction might be to print some data to the screen,
another instruction might be to calculate the result of an equation,
another to store the result in memory, and so on. Instructions are available
to allow you to enter data into the program, to perform a task over and
over again, counting, calculating and so on.

{\samepage \noindent \index{Hello World}
Here is an example of a very simple RTB program:
\begin{verbatim}
  10 PRINT "Hello, world"
  20 END
\end{verbatim}
What do you think it will do?}

\index{Line number}
The number 10 is called the line number. RTB doesn't
mind if numbers are missing -- the numbers are there for our benefit,
not the computers, so when we type programs into the computer, we can
use the line numbers to keep track of where we are, and if we forget to
enter a line, then we can use a number we're not already used in-between
the lines we need. So, here, if we wanted to print something more, we
can enter it in using line number 15 and the computer will insert it
in-between lines 10 and 20.

We can also use the line number to tell the computer to jump to a
different location in the program. So, if we were to enter this:
\begin{verbatim}
  15 GOTO 10
\end{verbatim}
then when we run the program, the computer will do exactly as its told to
do, it will execute line 10, printing {\tt Hello, world} to the screen,
then it would execute line 15 which tells it to jump back, or {\tt GOTO}\index{GOTO} to line 10
again. Computers are not clever\dots but they are very good at doing
exactly what they are told to do -- over and over again.

\noindent
This is our program so-far:
\begin{verbatim}
  10 PRINT "Hello, world"
  15 GOTO 10
  20 END
\end{verbatim}

This is probably the simplest program that you'll ever see. Variants
of the ``Hello, world'' program have been written for almost as long as
computers have existed.

In the next chapter we'll look at how we actually enter this program into
a computer running the RTB system and see how to make it work.
