\chapter{Variables}
\setcounter{ch-variables}{\value{chapter}}
\setcounter{ch-variables-page}{\value{page}}
\index{Variables}\index{Numbers}\index{Strings}

A variable is an area of computer memory that we can use to store data
in. It has a name associated with it as well as the memory locations
needed to store the data.

\section{Names}
\index{Variables!Names}
Variable names must start with a letter but may otherwise contain any
number of letters and digits and underscores. An example of a variable
name might be: {\tt counter}, or {\tt height} or {\tt numberOfCows}
and so on.

\section{Types}
\index{Variables!Types}
RTB Supports two types of variables; Numbers and Strings. These can
be scalars or arrays.

String variables are differentiated from numeric variables
by having the dollar character appended to them. E.g. {\tt name\$}
or {\tt homeTown\$} and so on.

A number is any decimal value like {\tt 1}, {\tt 3.14}, {\tt
-5}, and so on. Numbers may also be expressed in  scientific
notation. \index{Scientific Notation} and we use the letter ``e''
to represent the power of 10 so {\tt 1.234e7} is {\tt $1.234\times 10^7$},
or {\tt 12340000}. The number range is typically $\pm 10^{308}$.

A string is a sequence of characters, e.g. {\tt "Hello"}, {\tt
"Gordon"} and so on. Strings are always enclosed in double-quotes. The
maximum length of a string is limited by available memory.

\section{Assignment}
\index{Variables!Assignment}
We assign values to variables using the {\tt LET} instruction. e.g.
\index{LET}
\begin{verbatim}
  LET numberOfCows = 0
\end{verbatim}
and we can modify them as follows:
\begin{verbatim}
  LET numberOfCows = cowsInField + cowsInBarn
\end{verbatim}

The \meek {\tt LET} statement is optional.

\section{Numeric operators and Precedence}
\index{Numeric Operators}\index{Precedence}
There are various arithmetic and logical operators that can be
used with numbers.

In order of precedence, they are:
\begin{center}
\begin{tabular}[t]{|c|c|l|}
\hline
{\bf Precedence}	& {\bf Operator}	& {\bf Description}\\
\hline
\hline
{\bf Highest}	&\texttt{\^}	& Exponent, or ``raise to the power of''\\ 
\hline
		&\texttt{-}	& Unary minus\\
\hline
		&\texttt{NOT}	& Logical NOT\\
\hline
		&\texttt{*}\hspace{3mm} 
		\texttt{/}\hspace{3mm}
		\texttt{MOD} 
		\hspace{3mm}
		\texttt{DIV}	& Multiply, Divide, Modulo, Integer Division\\
\hline
		&\texttt{+}\hspace{3mm}
		\texttt{-}	& Addition, Subtraction\\
\hline
		&\texttt{|}\hspace{3mm}
		\texttt{\&}\hspace{3mm}
		\texttt{XOR}	& Logical OR, AND and XOR\\
\hline
		&\texttt{<}\hspace{3mm}
		\texttt{<=}\hspace{3mm}
		\texttt{>}\hspace{3mm}
		\texttt{>=}	& Conditional tests\\
\hline
		&\texttt{=}\hspace{3mm}
		\texttt{<>}	& Conditional Equals and Not Equals\\
\hline
{\bf Lowest}	&\texttt{AND}\hspace{3mm} 
		\texttt{OR}	& Conditional AND and OR\\
\hline
\end{tabular}
\end{center}

You can always use ()'s to alter the evaluation order, if required,
and in some cases they may help to make the code more readable and
obvious.

\section{String Operators}
There is only one string operator - the plus operator which we can use
to concatenate strings.
\begin{verbatim}
  firstName$ = "Gordon"
  lastName$ = "Henderson"
  fullName$ = firstName$ + " " + $lastName$
  PRINT fullName$
\end{verbatim}

\section{Arrays}
\index{Variables!Arrays}
Arrays must be declared before they are first used and we must know the
size of it before-hand. We declare them with the {\tt DIM}\footnote{Short
for Dimension.}\index{DIM} statement.

Arrays can be either numeric or string.  They can not hold mixed values.

Arrays can have more than one dimension - the limit to the number
of dimensions and overall size is program memory.

Arays are used as follows:
\begin{verbatim}
  10 DIM list (4)
  20 list(0) = 1
  30 list(2) = 4
  40 PRINT list(2) + list(0)
  50 END
\end{verbatim}

\section{Associative Arrays}
\index{Variables!Associative Arrays}\index{Variables!Map}
Associative arrays (sometimes called a {\em map}) is another way to
refer to the individial elements of an array. In the examples above we
used a number, however RTB also accepts a string.
\begin{verbatim}
  10 DIM record$ (10)
  20 record$ ("firstName") = "Gordon"
  30 record$ ("lastName") = "Henderson"
  40 record$ ("county") = "Devon"
  50 PRINT "First name is "; record$ ("firstName")
\end{verbatim}
Associative arrays can be multi-dimensional and you can freely mix
numbers and strings for the array indices.
